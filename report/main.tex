%%
%% This is file `sample-sigconf.tex',
%% generated with the docstrip utility.
%%
%% The original source files were:
%%
%% samples.dtx  (with options: `sigconf')
%% 
%% IMPORTANT NOTICE:
%% 
%% For the copyright see the source file.
%% 
%% Any modified versions of this file must be renamed
%% with new filenames distinct from sample-sigconf.tex.
%% 
%% For distribution of the original source see the terms
%% for copying and modification in the file samples.dtx.
%% 
%% This generated file may be distributed as long as the
%% original source files, as listed above, are part of the
%% same distribution. (The sources need not necessarily be
%% in the same archive or directory.)
%%
%%
%% Commands for TeXCount
%TC:macro \cite [option:text,text]
%TC:macro \citep [option:text,text]
%TC:macro \citet [option:text,text]
%TC:envir table 0 1
%TC:envir table* 0 1
%TC:envir tabular [ignore] word
%TC:envir displaymath 0 word
%TC:envir math 0 word
%TC:envir comment 0 0
%%
%%
%% The first command in your LaTeX source must be the \documentclass
%% command.
%%
%% For submission and review of your manuscript please change the
%% command to \documentclass[manuscript, screen, review]{acmart}.
%%
%% When submitting camera ready or to TAPS, please change the command
%% to \documentclass[sigconf]{acmart} or whichever template is required
%% for your publication.
%%
%%
\documentclass[sigconf]{acmart}
\usepackage{listings}
% \usepackage{minted}
\usepackage{color}
\usepackage{algorithm}
\usepackage{algorithmic}
\usepackage{multirow}
% \usepackage[table,xcdraw]{xcolor}

%New colors defined below
\definecolor{codegreen}{rgb}{0,0.6,0}
\definecolor{codegray}{rgb}{0.5,0.5,0.5}
\definecolor{codepurple}{rgb}{0.58,0,0.82}
\definecolor{backcolour}{rgb}{0.95,0.95,0.92}

% \lstdefinestyle{customstyle}{
%   backgroundcolor=\color{backcolour},
%   commentstyle=\color{codegreen},
%   keywordstyle=\color{magenta},
%   numberstyle=\tiny\color{codegray},
%   stringstyle=\color{codepurple},
%   basicstyle=\ttfamily,
%   % breakatwhitespace=false,
%   breaklines=false,
%   captionpos=b,
%   keepspaces=false,
%   numbers=left,
%   numbersep=3pt,
%   showspaces=false,
%   showstringspaces=false,
%   showtabs=false,
%   tabsize=2
% }

\lstdefinestyle{customstyle}{
  backgroundcolor=\color{backcolour}, commentstyle=\color{codegreen},
  keywordstyle=\color{magenta},
  numberstyle=\small\color{codegray},
  stringstyle=\color{codepurple},
  basicstyle=\small\ttfamily,
  breakatwhitespace=false,
  breaklines=true,
  captionpos=b,
  % columns=fullflexible,
  % keepspaces=false,
  numbers=left,
  numbersep=6pt,
  showspaces=false,
  showstringspaces=false,
  showtabs=false,
  tabsize=4,
  frame=single,
  xleftmargin=8.4pt,
  xrightmargin=6.4pt,
}

\lstset{basicstyle=\footnotesize,style=customstyle}

% \setminted{
% frame=lines,
% tabsize=2,
% breaklines	= true,
% xleftmargin	= 6pt,
% % baselinestretch=1.2,
% bgcolor=white,
% fontsize=\footnotesize,
% linenos
% }

%%
%% \BibTeX command to typeset BibTeX logo in the docs
\AtBeginDocument{%
  \providecommand\BibTeX{{%
        Bib\TeX}}}

% %% Rights management information.  This information is sent to you
% %% when you complete the rights form.  These commands have SAMPLE
% %% values in them; it is your responsibility as an author to replace
% %% the commands and values with those provided to you when you
% %% complete the rights form.
% \setcopyright{acmcopyright}
% \copyrightyear{2022}
% \acmYear{2022}
% \acmDOI{XXXXXXX.XXXXXXX}

% %% These commands are for a PROCEEDINGS abstract or paper.
% \acmConference[Conference acronym 'XX]{Make sure to enter the correct
%   conference title from your rights confirmation emai}{June 03--05,
%   2018}{Woodstock, NY}
% %%
% %%  Uncomment \acmBooktitle if the title of the proceedings is different
% %%  from ``Proceedings of ...''!
% %%
% %%\acmBooktitle{Woodstock '18: ACM Symposium on Neural Gaze Detection,
% %%  June 03--05, 2018, Woodstock, NY}
% \acmPrice{15.00}
% \acmISBN{978-1-4503-XXXX-X/18/06}

%%
%% Submission ID.
%% Use this when submitting an article to a sponsored event. You'll
%% receive a unique submission ID from the organizers
%% of the event, and this ID should be used as the parameter to this command.
%%\acmSubmissionID{123-A56-BU3}

%%
%% For managing citations, it is recommended to use bibliography
%% files in BibTeX format.
%%
%% You can then either use BibTeX with the ACM-Reference-Format style,
%% or BibLaTeX with the acmnumeric or acmauthoryear sytles, that include
%% support for advanced citation of software artefact from the
%% biblatex-software package, also separately available on CTAN.
%%
%% Look at the sample-*-biblatex.tex files for templates showcasing
%% the biblatex styles.
%%

%%
%% The majority of ACM publications use numbered citations and
%% references.  The command \citestyle{authoryear} switches to the
%% "author year" style.
%%
%% If you are preparing content for an event
%% sponsored by ACM SIGGRAPH, you must use the "author year" style of
%% citations and references.
%% Uncommenting
%% the next command will enable that style.
%%\citestyle{acmauthoryear}

\newcommand{\antreas}[1]{\textcolor{orange}{antreas: #1}}
\newcommand{\antonis}[1]{\textcolor{blue}{antonis: #1}}
\newcommand{\todo}[1]{\textcolor{red}{TODO: #1}}
\newcommand{\addedcontent}[1]{\textcolor{violet}{ADDITION: #1}}

%%
%% end of the preamble, start of the body of the document source.
\begin{document}

%%
%% The "title" command has an optional parameter,
%% allowing the author to define a "short title" to be used in page headers.
% \title{Rust: Compiler Enforced Memory Safety}
% \title{Rust's Memory Safety: An Attacker's Perspective}
\title{Building a generative AI-enabled chatbot for generating Fog/Edge
  Emulation deployments}

%%
%% The "author" command and its associated commands are used to define
%% the authors and their affiliations.
%% Of note is the shared affiliation of the first two authors, and the
%% "authornote" and "authornotemark" commands
%% used to denote shared contribution to the research.
\author{Antonis Louca}
\email{louca.antonis@ucy.ac.cy}
% \authornotemark[1]

\author{Panayiotis Papadopoulos}
\email{papadopoulos.m.panagiotis@ucy.ac.cy}
% \authornotemark[2]

\author{Andreas Chrisanthou}
\email{chrysanthou.m.andreas@ucy.ac.cy}
% \authornotemark[3]

%%
%% By default, the full list of authors will be used in the page
%% headers. Often, this list is too long, and will overlap
%% other information printed in the page headers. This command allows
%% the author to define a more concise list
%% of authors' names for this purpose.

%%
%% The abstract is a short summary of the work to be presented in the
%% article.

\begin{abstract}
  Developing a generative AI-powered chatbot tailored for Fog/Edge Emulation
deployments represents an exciting convergence of cutting-edge technology. This
endeavor draws inspiration from the capabilities of advanced Generative AI
models such as ChatGPT, which excel in comprehending and generating human-like
text. The primary objective here is to harness the potential of these models to
streamline and enrich the process of configuring Fog/Edge Emulation systems. To
achieve this, we design a user-friendly API that allows clients to
submit their requirements for a Fog Computing infrastructure. The system will
then extract relevant information from these inquiries and augment them with
the corresponding context. This augmentation will be achieved through the
implementation of prompt engineering techniques and in-context learning.
Subsequently, the system will transmit these enhanced queries to a powerful
large language model (LLM), such as the ChatGPT API, and relay the LLM's
responses back to the client. In this project we assume Fogify as the
underlying
emulation engine, and we create prompt engineering, using the  modeling
abstractions provided by Fogify's dedicated documentation page.

\end{abstract}

% \received{20 February 2007}
% \received[revised]{12 March 2009}
% \received[accepted]{5 June 2009}

%%
%% This command processes the author and affiliation and title
%% information and builds the first part of the formatted document.
\maketitle

\section{Introduction}
\label{sec:introduction}
The contemporary surge in the capabilities of generative AI models, exemplified
by GPT, has catalyzed significant advancements. This paper outlines the
development of an AI-powered chatbot tailored for Fog/Edge emulation, propelled
by the synergy of cutting-edge AI models and the critical role played by the
Fogify tool. Central to this initiative is the design of a user-friendly API
enabling clients to articulate Fog Computing infrastructure requirements.
Leveraging prompt engineering and contextual learning, our system refines user
queries, subsequently interfacing with a Large Language Model (LLM) like the
ChatGPT API. The incorporation of Fogify as the primary emulation tool, guided
by its documentation, ensures optimal outcomes during prompt engineering. This
paper highlights the integration of AI technologies for streamlining Fog/Edge
emulation processes.
\section{Background}
\label{sec:background}

\section{methodology}
\label{sec:methodology}

\subsection{Terminology}

Chroma DB is an open-source vector store, which provides the tools for building
a knowledge base for LLMs

OpenAi: A service that allows developers to integrate advanced natural language
processing capabilities into their applications

Fogify	is an emulation Framework for modelling, deploying and experimenting
with fog testbeds. Provides a toolset to model complex fog topologies

Python is a high-level programming language known for its readability and
simplicity

LangChain is a framework designed to simplify creation of applications using
large language models (LLMs)

\subsection{Process we followed}
A diagram of a chat

Description automatically generatedWe started our process by extracting the
Fogify documentation. More precisely we read all the documents with extension
.xml, .html, .md, yaml. Then those documents are segmented into chunks of 1000
characters. All those chunks form a list of document chunks.Then we need to
transform this list into vectors. We achieve that by utilizing OpenAI in order
to convert chunks to vector embeddings, those embeddings are stored in Chroma
DB which is used for the persistent storage of the embeddings.After the above
steps our tool is ready for use. More precisely, when the chat receives a user
query, it searches the knowledge base in order to receive most relevant
information (context) the we perform prompt engineering  ( context, user's
question, memory of recent conversation messages between user and AI Model) and
feed the prompy to AI model to get response
\section{Implementation}
\label{sec:implementation}
\section{Evaluation}
\label{sec:evaluation}
In this section we evaluate the chatbot tool we created for this project. We
use a set of questions to evaluate the performance of the chatbot. For question
we provide the best response from our chatbot along with a piece of context we
have from the documentation of the Fogify tool. This way we compare the results
of the chatbot in comparison with the context of the documentation. Below we
provide the set of questions we asked our chatbot.

\subsubsection*{Questions used in the evaluation stage}
\begin{enumerate}
    \item What is an action and how can i add an action using python code?
    \item What are some parameters that we specify in the Fogify emulator for
          network connectivity? Can you give me an example of yaml
          configuration?
    \item What are blueprints and what are their different variables? Can you
          also give me an example?
    \item How does Fofigy compose the network latency can you give me an
          example of configuration with the different variables in that yaml?
    \item Explain the different parameters of the blueprint.
    \item What are the performance metrics captured by Fogify. Can you give me
          the metric and a small description for it?
    \item Can a user add its own metrics to fogify? How can a user do that can
          you give me an example?
    \item How can someone interact with the emulated testbed?
    \item Can you give me an example in python code on how i can use the fogify
          SDK library to interact with the Fogify emulation tool?
\end{enumerate}


\section{Future work}
\label{sec:FW}

Future work for this project involves the improvement of all pieces that make
up our implementation.

The most obvious candidate is the vector store. Since the LLM provides
responses based on the information we give to it, the better the information
that we give to it the better the responses will be. As such, we can try to
improve the process of storing the documentation in our vector store. This
includes the breaking of files into smaller pieces, removing irrelevant data
from our files such as HTML tags, metadata, or other data that is not
information such as HTML tags for buttons.

Breaking up our files into smaller pieces can also help us in providing more
detailed information. We can also add to our context more information instead
of the most relevant from our query. WE can give it the top three results from
our query.

Finally, we can try and improve our queries into the vector store. We might
have relevant information in our vector store that isn’t brought up because our
query isn’t actually the best for what we are looking for.

The second part of our implementation that we can look into improving is our
LLM. We use a pre-trained model that is provided by OpenAI, this is because
designing and implementing a LLM model is not a easy task. There are multiple
other models that we can use, and we can use these other models to make
comparisons.

The most effective approach would be to use a publicly available model and
train it to better understand the information we give to it. This requires a
lot of time and data not to mention knowledge of Machine Learning.

\section{Conclusions}
\label{sec:conclusions}

%%
%%
%% The next two lines define the bibliography style to be used, and
%% the bibliography file.
\bibliographystyle{ACM-Reference-Format}
\bibliography{bibliography}
\end{document}
\endinput
